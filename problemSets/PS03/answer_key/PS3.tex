\documentclass[12pt,letterpaper]{article}
\usepackage{graphicx,textcomp}
\usepackage{natbib}
\usepackage{setspace}
\usepackage{fullpage}
\usepackage{color}
\usepackage[reqno]{amsmath}
\usepackage{amsthm}
\usepackage{fancyvrb}
\usepackage{amssymb,enumerate}
\usepackage[all]{xy}
\usepackage{endnotes}
\usepackage{lscape}
\newtheorem{com}{Comment}
\usepackage{float}
\usepackage{hyperref}
\newtheorem{lem} {Lemma}
\newtheorem{prop}{Proposition}
\newtheorem{thm}{Theorem}
\newtheorem{defn}{Definition}
\newtheorem{cor}{Corollary}
\newtheorem{obs}{Observation}
\usepackage[compact]{titlesec}
\usepackage{dcolumn}
\usepackage{tikz}
\usetikzlibrary{arrows}
\usepackage{multirow}
\usepackage{xcolor}
\newcolumntype{.}{D{.}{.}{-1}}
\newcolumntype{d}[1]{D{.}{.}{#1}}
\definecolor{light-gray}{gray}{0.65}
\usepackage{url}
\usepackage{listings}
\usepackage{color}

\definecolor{codegreen}{rgb}{0,0.6,0}
\definecolor{codegray}{rgb}{0.5,0.5,0.5}
\definecolor{codepurple}{rgb}{0.58,0,0.82}
\definecolor{backcolour}{rgb}{0.95,0.95,0.92}

\lstdefinestyle{mystyle}{
	backgroundcolor=\color{backcolour},   
	commentstyle=\color{codegreen},
	keywordstyle=\color{magenta},
	numberstyle=\tiny\color{codegray},
	stringstyle=\color{codepurple},
	basicstyle=\footnotesize,
	breakatwhitespace=false,         
	breaklines=true,                 
	captionpos=b,                    
	keepspaces=true,                 
	numbers=left,                    
	numbersep=5pt,                  
	showspaces=false,                
	showstringspaces=false,
	showtabs=false,                  
	tabsize=2
}
\lstset{style=mystyle}
\newcommand{\Sref}[1]{Section~\ref{#1}}
\newtheorem{hyp}{Hypothesis}

\title{Problem Set 3}
\date{Due: November 12, 2021}
\author{Applied Stats/Quant Methods 1}


\begin{document}
	\maketitle
	\section*{Instructions}
	\begin{itemize}
		\item Please show your work! You may lose points by simply writing in the answer. If the problem requires you to execute commands in \texttt{R}, please include the code you used to get your answers. Please also include the \texttt{.R} file that contains your code. If you are not sure if work needs to be shown for a particular problem, please ask.
		\item Your homework should be submitted electronically on GitHub in \texttt{.pdf} form.
		\item This problem set is due before class on Friday November 12, 2021. No late assignments will be accepted.
		\item Total available points for this homework is 80.
	\end{itemize}

	
		\vspace{.25cm}
	
\noindent In this problem set, you will run several regressions and create an add variable plot (see the lecture slides) in \texttt{R} using the \texttt{incumbents\_subset.csv} dataset. Include all of your code.

	\vspace{.5cm}
\section*{Question 1} %(20 points)}
\vspace{.25cm}
\noindent We are interested in knowing how the difference in campaign spending between incumbent and challenger affects the incumbent's vote share. 
	\begin{enumerate}
		\item Run a regression where the outcome variable is \texttt{voteshare} and the explanatory variable is \texttt{difflog}.	\vspace{5cm}
		
		Create object votesharedifflog from regression on voteshre and difflog
		
		Use lm function. Code is as follows....
		
	- votesharedifflog <- lm(incumbents dollar sign voteshare ~ incumbents dollar sign difflog)
	
	- votesharedifflog 
	
	Intercept = 0.58, slope = 0.04
		
	
	
		\item Make a scatterplot of the two variables and add the regression line. 	\vspace{7cm}
	
		Use plot() and abline functions in R
		
		- plot(votesharedifflog)
		
		- abline(votesharedifflog)
		
	Moderately positive/null relation observed in plot
		
		\item Save the residuals of the model in a separate object.	\vspace{7cm}
		
		Use residuals() function in R on votesharedifflog
		
	- 	residuals1 <- residuals(votesharedifflog) 
	
	Print residuals1
	
	- 	residuals1
		\item Write the prediction equation.
		
		Equation = intercept + slope multiplyed by random x value
		
		E.g. use 15 as random x value
		
	- 	0.58 + 0.04*15
		
		Predicted value is 1.18
	\end{enumerate}
	
\newpage

\section*{Question 2}% (20 points)}
\noindent We are interested in knowing how the difference between incumbent and challenger's spending and the vote share of the presidential candidate of the incumbent's party are related.	\vspace{.25cm}
	\begin{enumerate}
		\item Run a regression where the outcome variable is \texttt{presvote} and the explanatory variable is \texttt{difflog}.	\vspace{5cm}
		
		Create object difflogpresvote from regression of presvote and difflog
		
		Use lm function for regression
		
	-	difflogpresvote <- lm(incumbents$presvote ~ incumbents$difflog)
		
	Print difflogpresvote 
	 
	 Intercept = 0.51, slope = 0.02
	
		\item Make a scatterplot of the two variables and add the regression line. 	\vspace{5cm}
		
		Make scatterplot of difflogpresvote and add regression line using plot and abline functions in R
		
	-	plot(difflogpresvote)
	
	-	abline(difflogpresvote)
		
		Moderately positive/null relation observed in plot
		
		\item Save the residuals of the model in a separate object.	\vspace{5cm}
		
		 Save residuals of model in sperate object residuals2
				
	- 	residuals2 <- residuals(difflogpresvote)
	
		Print residuals2
		- residuals2
		\item Write the prediction equation.
	
	Equation = intercept + slope multiplyed by random x value
	
	E.g. use 5 as random x value
	
	-	0.51 + 0.02*5
	
	Predicted y value is 0.61
	\end{enumerate}
	
	\newpage	
\section*{Question 3}% (20 points)}

\noindent We are interested in knowing how the vote share of the presidential candidate of the incumbent's party is associated with the incumbent's electoral success.
	\vspace{.25cm}
	\begin{enumerate}
		\item Run a regression where the outcome variable is \texttt{voteshare} and the explanatory variable is \texttt{presvote}.
			\vspace{5cm}
			
		Create object presvotevoteshare from regression on presvote and voteshare
		
		Use lm function for regression in R
		
		- presvotevoteshare <- lm(incumbents$presvote ~ incumbents$voteshare)
	
	Print presvotevoteshare	
	
	-	presvotevoteshare	
	
		\item Make a scatterplot of the two variables and add the regression line. 
			\vspace{5cm}
			
			Use plot and abline functions in R
	
		-	plot(presvotevoteshare)
		
		-	abline(presvotevoteshare)
		
		Moderately positive/null relation observed in plot
		
		\item Write the prediction equation.
		
		Equation is y = 0.2036 (intercept) + 0.5304 (slope) * random x 
		
		E.g.  X value = 9
		 
		- 0.2036 + 0.5304*9
		
		Predicted y value = 4.9772
		
	\end{enumerate}
	

\newpage	
\section*{Question 4}% (20 points)}
\noindent The residuals from part (a) tell us how much of the variation in \texttt{voteshare} is $not$ explained by the difference in spending between incumbent and challenger. The residuals in part (b) tell us how much of the variation in \texttt{presvote} is $not$ explained by the difference in spending between incumbent and challenger in the district.
	\begin{enumerate}
		\item Run a regression where the outcome variable is the residuals from Question 1 and the explanatory variable is the residuals from Question 2.	\vspace{6cm}
		
		Run regression between outcome var residuals1 and explanatory var residuals2
		
		Create object combresiduals from regression
		
		Code is as follows...
		
	- combresiduals <- lm(residuals1 ~ residuals2)
	
	Print combresiduals
	
	-	combresiduals 
	
	Intercept = 4.498, slope = 6.866
		
		\item Make a scatterplot of the two residuals and add the regression line. 	\vspace{6cm}
		
		Use plot and abline functions on combresiduals
		
	-	plot(combresiduals)
		
	-	abline(combresiduals)
	
	Strongly positive relation between residuals indicated by plot
	
		\item Write the prediction equation.
		
		Add intercept and slope and multipy slope by random x value (i.e. residual2 value)
		
		E.g. X value = 20
		
		- 4.498 + 6.866*20
	
	Predicted y value when x is 20 = 141.818
	
\end{enumerate}
	
	\newpage	

\section*{Question 5}% (20 points)}
\noindent What if the incumbent's vote share is affected by both the president's popularity and the difference in spending between incumbent and challenger? 
	\begin{enumerate}
		\item Run a regression where the outcome variable is the incumbent's \texttt{voteshare} and the explanatory variables are \texttt{difflog} and \texttt{presvote}.	\vspace{5cm}
		
		Multi-Variate Regression 
	
	Create object votesharediffpresvote from regression. 
	
	Use lm function. Use + sign to add on a second predictor (or x) variable
	
- votesharediffpresvote <- lm(voteshare ~ difflog+presvote, data=incumbents)

Print votesharediffpresvote

-	votesharediffpresvote 

Intercept = 0.44864, slopes = 0.03554 and 0.25688

		\item Write the prediction equation.	\vspace{5cm}
		
		Use x values from sample. E.g. First difflog value 0.570 and first presvote value 0.527

	-	0.459+0.036*0.570+0.257*0.527
	
	Predicted y value = 0.615
		
		\item What is it in this output that is identical to the output in Question 4? Why do you think this is the case?%	\vspace{5cm}
	%	\item Reflect on your finding. Don't write anything. Just think about it.
	\end{enumerate}




\end{document}
